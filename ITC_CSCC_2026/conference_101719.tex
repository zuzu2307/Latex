\documentclass[conference]{IEEEtran}
\IEEEoverridecommandlockouts
% The preceding line is only needed to identify funding in the first footnote. If that is unneeded, please comment it out.
\usepackage{cite}
\usepackage{amsmath,amssymb,amsfonts}
\usepackage{algorithmic}
\usepackage{graphicx}
\usepackage{textcomp}
\usepackage{xcolor}
\usepackage{multirow}
\usepackage{booktabs}
\usepackage{adjustbox}
\def\BibTeX{{\rm B\kern-.05em{\sc i\kern-.025em b}\kern-.08em
    T\kern-.1667em\lower.7ex\hbox{E}\kern-.125emX}}
\begin{document}

\title{Enhancing Perception of Recessed Features in VR Haptics through Vibrotactile Attenuation\\
% {\footnotesize \textsuperscript{*}Note: Sub-titles are not captured in Xplore and
% should not be used}
% \thanks{Identify applicable funding agency here. If none, delete this.}
}

\author{\IEEEauthorblockN{1\textsuperscript{st} Korntawat Witchuvanit}
\IEEEauthorblockA{\textit{Dept. of Intelligent Information Systems Engineering} \\
\textit{Fukuoka Institute of Technology}\\
Fukuoka, Japan \\
bd25201@bene.fit.ac.jp}
\and
\IEEEauthorblockN{2\textsuperscript{nd} Makio Ishihara}
\IEEEauthorblockA{\textit{Faculty of Information Engineering} \\
\textit{Fukuoka Institute of Technology}\\
Fukuoka, Japan \\
m-ishihara@bene.fit.ac.jp}
% \and
% \IEEEauthorblockN{3\textsuperscript{rd} Given Name Surname}
% \IEEEauthorblockA{\textit{dept. name of organization (of Aff.)} \\
% \textit{name of organization (of Aff.)}\\
% City, Country \\
% email address or ORCID}
% \and
% \IEEEauthorblockN{4\textsuperscript{th} Given Name Surname}
% \IEEEauthorblockA{\textit{dept. name of organization (of Aff.)} \\
% \textit{name of organization (of Aff.)}\\
% City, Country \\
% email address or ORCID}
% \and
% \IEEEauthorblockN{5\textsuperscript{th} Given Name Surname}
% \IEEEauthorblockA{\textit{dept. name of organization (of Aff.)} \\
% \textit{name of organization (of Aff.)}\\
% City, Country \\
% email address or ORCID}
% \and
% \IEEEauthorblockN{6\textsuperscript{th} Given Name Surname}
% \IEEEauthorblockA{\textit{dept. name of organization (of Aff.)} \\
% \textit{name of organization (of Aff.)}\\
% City, Country \\
% email address or ORCID}
}

\maketitle

\begin{abstract}
Realistic texture perception poses a significant challenge in Virtual Reality (VR) haptics, particularly when it comes to representing subtle geometric variations, such as surface dents or holes. Traditionally, VR haptic systems employ vibrotactile intensification, which increases the vibration frequency when a virtual fingertip makes contact with a raised feature or edge. However, the effectiveness of this method for conveying recessed features remains unclear. We propose a vibrotactile attenuation strategy, which reduces the vibration frequency when the user's virtual fingertip interacts with a recessed area. Our findings indicate that this reduction in frequency effectively helps participants perceive the presence of a recessed surface.
\end{abstract}

\begin{IEEEkeywords}
Virtual Reality, Haptic, Texture Rendering, Vibration Feedback
\end{IEEEkeywords}

\section{Introduction}
Achieving realistic physical interactions in Virtual Reality (VR) relies heavily on accurately perceiving texture and surface geometry. While visual and auditory fidelity have advanced significantly, the sense of touch often falls short, limiting the effectiveness of immersive applications such as surgical training and virtual prototyping \cite{10.1146/annurev-control-060117-105043}. Current haptic technologies primarily focus on vibrotactile enhancement, which involves increasing the frequency or amplitude of vibrations to simulate raised features like bumps and edges \cite{10.1162/1054746041944867}. While this method is effective for conveying positive geometric variations, it struggles to intuitively communicate subtle recessed features, such as dents or holes. This absence of a distinct sensory cue for depression creates a significant gap in the realism of virtual tactile feedback \cite{10.3758/APP.71.7.1439}.
To address the uncertainty regarding tactile feedback, we propose a vibrotactile attenuation strategy that connects recessed features with a decrease in frequency. Our experimental framework consists of three distinct conditions. We establish a baseline frequency of 150 Hz for flat surfaces and employed a 250 Hz "Strong Frequency" for peaks. Notably, we introduce a 50 Hz "Low Frequency" condition for recessed areas to simulate the sensation of contact loss. This study compares these attenuation and intensification strategies to evaluate whether reducing frequency provides a more intuitive representation of concave geometries compared to traditional methods.

\section{Methodology}
To evaluate the effectiveness of vibrotactile attenuation, we develop a custom haptic interface integrated with a virtual environment. The hardware system utilizes a Meta Quest 3 Head-Mounted Display (HMD) for immersive visualization and high-fidelity hand tracking. For tactile delivery, a single Eccentric Rotating Mass (ERM) coin-type motor is secured to the participant's index finger and driven by an ESP32 microcontroller (Fig. 1). The virtual environment features a large, flat gray surface and includes three types of surfaces: the Base Surface (flat region), the Recess (hole), and the Peak (bump). 

\begin{figure}[htbp]
\centerline{\includegraphics[width=0.25\textwidth]{Figure/Device.jpg}}
\caption{The Haptic Device.}
\label{Fig_1}
\end{figure}

The microcontroller is programmed to instantly adjust the motor frequency based on the virtual fingertip's position relative to the surface. This approach aims at stimulating specific mechanoreceptors, enabling realistic texture perception by utilizing distinct frequency tiers. The control logic is implemented as follows:
\begin{itemize}
    \item Base Surface (150 Hz): When contact is made with a flat area, the motor activates a mid-range frequency to indicate the smooth surface
    \item Recess (50 Hz): When entering a depression, the system adjusts the vibration frequency to 50 Hz to provide vibrotactile attenuation. 
    \item Peak (250 Hz): When entering a raised bump, the system increases the vibration frequency to 250 Hz for vibrotactile intensification.
\end{itemize}

\section{Procedure}
A total of 13 students studying abroad (12 males and 1 female, aged 22 to 27 years) are recruited from the university campus. The experiment follows a systematic five-step procedure to ensure participants could accurately distinguish between the vibrotactile cues:

\begin{enumerate}
    \item Participants attach all the equipment
    \item Participants are introduced to the three states of vibration indicated by the different colors of the sphere that appear based on the state assigned to them (Fig. 2). 
    \begin{itemize}
        \item When a participant touches the gray surface, the vibration is set to a base surface vibration of 150 Hz (Fig. 2a). 
        \item When the hand touches the red sphere, the vibration changes to 50 Hz, representing a recess in the surface (Fig. 2b).
        \item When the hand touches the blue sphere, the vibration shifts to 250 Hz, representing a peak on the surface (Fig. 2c).
    \end{itemize}

    \begin{figure}[htbp]
        \centerline{\includegraphics[width=0.5\textwidth]{Figure/3_surface.png}}
        \caption{(a) Base Surface, (b) Recess Surface, (c) Peak Surface}
        \label{Fig_2}
    \end{figure}

    \item After introducing the three states of vibration, the experiment commences. During the actual experiment, participants cannot see any colored spheres. They move their fingers to touch and swipe across the surface, then respond to whether they feel any holes on the surface. If they do not feel any holes, the surface will either have peaks or be flat.
    \item The experiment consists of 30 trials (3 sets; 10 trials per set; 3-minute breaks between each set). There are 10 base surfaces, 10 recess surfaces, and 10 peak surfaces, presented in a random order.
    \item After completing 30 trials, each participant fills out 8 questionnaires, using a 7-point Likert scale. Each two questions are grouped into one of four- categories. 
\end{enumerate}

\section{Result}
The experiment showcased nearly perfect classification performance in distinguishing the recess surface condition from other surfaces. Table 1 presents the confusion matrix summarizing participants' responses on whether the surface they touched had any holes. The results indicated that participants were able to effectively differentiate the recess surface from the others, demonstrating exceptional performance and reliability across the board. This high accuracy was evidenced by a Recall score of 1.0, a strong Precision of 0.963, and a Specificity of 0.981.

\begin{table}[h]
\centering
\caption{Participants Answer Confusion Matrix}
\begin{tabular}{|l|c|c|}
\hline
\multirow{2}{*}{Participants' Answer} & \multicolumn{2}{c|}{Actual Answer} \\ \cline{2-3}
 & Recess Surface & Other Surfaces \\ \hline
Feeling holes on surface & 130 & 5 \\ \hline
Not feeling any holes on surface & 0 & 255 \\ \hline
\end{tabular}
\end{table}

The results of the questionnaire were analyzed using Wilcoxon signed-rank tests to determine whether participants' perceptual ratings significantly differed from the neutral midpoint (4) of the Likert scale. For all categories, the findings consistently demonstrated strong support for the proposed attenuation strategy, as shown in Table 2.

Participants reported a significantly enhanced perception of holes ($p = 2.98 \times 10^{-8}$), peaks ($p = 2.98 \times 10^{-8}$), and surface differences ($p = 2.98 \times 10^{-8}$), all of which were significantly higher than the neutral reference level. Additionally, overall immersion was also rated significantly above neutral ($p = 2.98 \times 10^{-8}$).

% \begin{table}[h]
% \centering
% \caption{Wilcoxon Signed-Rank Test Results for Questionnaire Categories}

% \begin{tabular}{p{2cm}ccccp{2cm}}
% \toprule
% Category & N & Median & $p$-value & Sig. & Conclusion \\
% \midrule

% Feeling Holes 
% & 26 & 5.0 & $2.98\times10^{-8}$ & *** & Significantly higher than Neutral. \\

% Feeling Peaks 
% & 26 & 6.0 & $2.98\times10^{-8}$ & *** & Significantly higher than Neutral. \\

% Feeling Surface’s Difference 
% & 26 & 6.0 & $2.98\times10^{-8}$ & *** & Significantly higher than Neutral. \\

% Immersion 
% & 26 & 5.0 & $2.98\times10^{-8}$ & *** & Significantly higher than Neutral. \\
% \bottomrule
% \end{tabular}

% \vspace{4pt}
% \footnotesize{Note: Significance levels: *$p<0.05$, **$p<0.01$, ***$p<0.001$.}

% \end{table}

\begin{table}[h]
\centering
\caption{Wilcoxon Signed-Rank Test Results for Questionnaire Categories}
\renewcommand{\arraystretch}{1.4}
\begin{tabular}{|p{2cm}|c|c|c|c|p{2cm}|}
\hline
Category & N & Median & $p$-value & Sig. & Conclusion \\ \hline

Feeling Holes 
& 26 & 5.0 & $2.98 \times 10^{-8}$ & *** 
& Significantly higher than Neutral. \\ \hline

Feeling Peaks 
& 26 & 6.0 & $2.98 \times 10^{-8}$ & *** 
& Significantly higher than Neutral. \\ \hline

Feeling Surface’s Difference 
& 26 & 6.0 & $2.98 \times 10^{-8}$ & *** 
& Significantly higher than Neutral. \\ \hline

Immersion 
& 26 & 5.0 & $2.98 \times 10^{-8}$ & *** 
& Significantly higher than Neutral. \\ \hline

\end{tabular}

\vspace{4pt}
\footnotesize{Note: Significance levels: *$p<0.05$, **$p<0.01$, ***$p<0.001$.}
\end{table}

\section*{DISCUSSION AND CONCLUSION}

The experimental results confirmed the effectiveness of the vibrotactile attenuation strategy, achieving nearly perfect classification for recessed features, with a Recall of 1.0 and a Specificity of 0.981. By reducing the frequency to 50 Hz, the system established a clear sensory boundary, eliminating the confusion often associated with traditional intensification methods. These metrics indicated a clear and distinct perceptual boundary between recess and non-recess surfaces.
The results of the questionnaire supported this finding, as Wilcoxon signed-rank tests showed significant perceptual differences across all evaluated categories. Participants reported a substantially enhanced perception of holes, peaks, and surface differences, along with heightened immersion, with all ratings significantly surpassing the neutral midpoint. 
In conclusion, this study showed that mapping surface depressions to a low-frequency (50 Hz) cue created a more intuitive sensory experience compared to traditional methods. By moving beyond the typical "intensification-only" approach, this research offers a more nuanced sensory cue for haptic interactions. In the future, we plan to explore the integration of this attenuation strategy with variable indentation depths and different material textures to enhance the realism of virtual tactile experiences.




\bibliographystyle{IEEEtran}
\bibliography{Ref}
% \begin{thebibliography}{00}
% \bibitem{b1} G. Eason, B. Noble, and I. N. Sneddon, ``On certain integrals of Lipschitz-Hankel type involving products of Bessel functions,'' Phil. Trans. Roy. Soc. London, vol. A247, pp. 529--551, April 1955.
% \bibitem{b2} J. Clerk Maxwell, A Treatise on Electricity and Magnetism, 3rd ed., vol. 2. Oxford: Clarendon, 1892, pp.68--73.
% \bibitem{b3} I. S. Jacobs and C. P. Bean, ``Fine particles, thin films and exchange anisotropy,'' in Magnetism, vol. III, G. T. Rado and H. Suhl, Eds. New York: Academic, 1963, pp. 271--350.
% \bibitem{b4} K. Elissa, ``Title of paper if known,'' unpublished.
% \bibitem{b5} R. Nicole, ``Title of paper with only first word capitalized,'' J. Name Stand. Abbrev., in press.
% \bibitem{b6} Y. Yorozu, M. Hirano, K. Oka, and Y. Tagawa, ``Electron spectroscopy studies on magneto-optical media and plastic substrate interface,'' IEEE Transl. J. Magn. Japan, vol. 2, pp. 740--741, August 1987 [Digests 9th Annual Conf. Magnetics Japan, p. 301, 1982].
% \bibitem{b7} M. Young, The Technical Writer's Handbook. Mill Valley, CA: University Science, 1989.
% \end{thebibliography}
% \vspace{12pt}
% \color{red}
% IEEE conference templates contain guidance text for composing and formatting conference papers. Please ensure that all template text is removed from your conference paper prior to submission to the conference. Failure to remove the template text from your paper may result in your paper not being published.

\end{document}
