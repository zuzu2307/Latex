% Chapter 1

\chapter{Introduction} % Main chapter title

\label{Chapter1} % For referencing the chapter elsewhere, use \ref{Chapter1} 

%----------------------------------------------------------------------------------------

% Define some commands to keep the formatting separated from the content 
\newcommand{\keyword}[1]{\textbf{#1}}
\newcommand{\tabhead}[1]{\textbf{#1}}
\newcommand{\code}[1]{\texttt{#1}}
\newcommand{\file}[1]{\texttt{\bfseries#1}}
\newcommand{\option}[1]{\texttt{\itshape#1}}

%----------------------------------------------------------------------------------------
This chapter presents backgrounds, objective, and the structure of this thesis.

\section{Backgrounds}
Virtual reality (VR) has evolved into a powerful platform that provides immersive experiences across various applications, including gaming, training simulations, healthcare, education, and therapy\cite{10.3389/frobt.2016.00074}. By integrating visual, auditory, and interactive elements, VR environments create a strong sense of presence, which significantly enhances user engagement and the perception of realism\cite{10.5555/207922}. Considerable advancements in display technologies, accurate motion tracking systems, and spatial audio reproduction have greatly improved user immersion. However, despite these technological advancements, achieving realistic tactile feedback remains a significant challenge, limiting the overall sensory fidelity and immersion in VR applications\cite{10.1146/annurev-control-060117-105043}.

To reduce the sensory gap, haptic feedback plays a vital role by providing tactile stimuli that replicate real-world touch sensations\cite{10.1146/annurev-control-060117-105043}. Various haptic devices, ranging from basic vibrotactile actuators to advanced force-feedback systems and wearable gloves, have been thoroughly explored to enhance tactile realism. While handheld controllers are commonly used due to their simplicity and practicality, they often restrict natural hand movements and hinder intuitive interactions with virtual objects, which can lead to a diminished sense of immersion. In contrast, wearable haptic solutions that directly integrate with the user's body present promising alternatives, allowing for natural and intuitive hand gestures and interactions\cite{7922602}. However, creating wearable devices that balance lightweight design, responsiveness, ergonomic comfort, and adaptability continues to be a significant challenge within Human-Computer Interaction (HCI) research.

To address these limitations, I propose the design and implementation of a custom wearable glove embedded with flex sensors that capture precise hand gestures, a Motion Processing Unit (MPU) for accurate movement tracking, and a coin motor to provide detailed haptic feedback. This glove has been specifically engineered to enhance the perception of virtual textures by utilizing vibrational feedback across three distinct levels of granularity, carefully chosen for their effectiveness in conveying realistic texture sensations\cite{10.1145/3025453.3025812}. The main contributions of this research include the successful development of a responsive and lightweight wearable haptic device, as well as a comprehensive empirical evaluation of user tactile perception across different vibration granularities. This work significantly advances the realism and immersive quality of haptic feedback systems in virtual reality.

%----------------------------------------------------------------------------------------

\section{Objective}

The objective of this research is to design and evaluate a lightweight and responsive wearable haptic glove that incorporates flex sensors, a Motion Processing Unit (MPU), and a coin motor. This glove aims to enhance the perception of realistic textures in virtual environments. By providing vibration feedback at three different levels of granularity, this study empirically assesses users' ability to accurately distinguish virtual textures based on perceived smoothness. Ultimately, this research seeks to advance tactile realism and improve overall immersion in virtual reality applications.

\section{Thesis structure}

This thesis consists of six chapters. Chapter~\ref{Chapter1}  introduces the background, objectives, and structure of the thesis. Chapter~\ref{Chapter2} provides a comprehensive review of related work in the field. Chapter~\ref{Chapter3} describes the design and implementation of a wearable haptic device that incorporates flex sensors, a coin motor, and an MPU. Chapter~\ref{Chapter4} outlines the experimental methodology and assesses the ability of users to distinguish various texture granularities. Chapter~\ref{Chapter5} presents the results and discusses practical considerations and potential real-world applications of the proposed device. Finally, Chapter~\ref{Chapter6} concludes the thesis and suggests directions for future research.



