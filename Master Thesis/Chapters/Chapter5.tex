\chapter{Result and Discussion} % Main chapter title

\label{Chapter5} % Change X to a consecutive number; for referencing this chapter elsewhere, use \ref{ChapterX}


The following sections outline the experimental evaluation of the proposed system, which consists of two main experiments and a subjective evaluation conducted afterward. The first experiment assesses participants' ability to differentiate between three distinct cycle rates, while the second experiment investigates how these cycle rates affect the perceived realism of various virtual textures. Both experiments utilize objective classification accuracy as well as subjective evaluation scores to validate the effectiveness of the proposed method.

%----------------------------------------------------------------------------------------
%	SECTION 1
%----------------------------------------------------------------------------------------
\section{Experiment 1's Results and Discussion}
The performance of participants in identifying the three vibration cycle rates—0.2, 0.5, and 1.0—is summarized in Fig.~\ref{fig:ex1_results} and Fig.~\ref{fig:C_W_ex1}, which shows the overall classification accuracy across all trials. Each cycle rate had a total of 30 trials (six participants with five repetitions each). The highest classification accuracy was recorded at the 1.0 cycle rate, where participants correctly identified the vibration pattern in 26 out of 30 trials, resulting in an accuracy of 86.7\%. This suggests that higher-frequency vibration rhythms with shorter pauses are more easily perceived and classified by users, likely due to their more continuous and distinct tactile sensation.

\begin{figure}[H]\centering
	\includesvg[width=1.0\textwidth, inkscapelatex=true]{Pictures/ex1_result}
	\caption{Participants' accuracy in identifying different vibration cycle rates.}\label{fig:ex1_results}
\end{figure}

\begin{figure}[H]\centering
	\includesvg[width=0.8\textwidth, inkscapelatex=true]{Pictures/C_W_Chart}
	\caption{Participants' correct and wrong answers in identifying different vibration cycle rates.}\label{fig:C_W_ex1}
\end{figure}


At a cycle rate of 0.5, accuracy slightly decreased to 24 out of 30 trials (80\%). The lowest accuracy was observed at a cycle rate of 0.2, with participants achieving 23 out of 30 correct identifications (76.7\%). These results indicate a trend in which participant accuracy declines as the vibration rhythm slows down, leading to a more intermittent tactile sensation. %This finding aligns with previous research in haptic perception, which has shown that higher-frequency or more intense vibrotactile cues are generally easier to detect and discriminate due to stronger stimulation of tactile receptors.

The moderate reduction in accuracy at lower cycle rates may be attributed to the diminished temporal density of tactile feedback, making it more difficult for participants to form clear perceptual distinctions, especially when the vibration intervals become more sparse. Notably, even at the lowest cycle rate, participants maintained a classification rate well above chance level (33.3\%), indicating that the PWM cycle rate modulation approach provides reliably distinguishable tactile patterns.

Overall, these findings validate the effectiveness of the proposed vibrotactile feedback method, which utilizes pulse-width modulation to control vibration cycle rates, in delivering perceivable differences in vibrotactile rhythms. The results demonstrate that participants can accurately classify distinct cycle rates, particularly in the higher rhythm ranges, supporting the system’s suitability for conveying virtual texture roughness and bumpiness through simple vibration control strategies.

%%%%%%%%%%%%%%%%%%%%%%%%%%%%%%%%%%%%%%
\section{Experiment 2's Results and Discussion}
Participants' preferences for associating vibration cycle rates with the virtual textures—brick, grass, and marble—are illustrated in Fig.~\ref{fig:ex2_results}. The distribution of selections demonstrates how participants mapped specific vibration rhythms to different surface types. For the brick texture, responses were equally divided between 0.2 (low cycle rate) and 0.5 (medium cycle rate), with three participants choosing each option. This split suggests that participants found both slower and moderate rhythmic vibrations to be appropriate representations of the rough and uneven characteristics of brick surfaces. Notably, no participants selected the 1.0, indicating a shared perception that fast, continuous vibrations were not suitable for representing rough textures.

\begin{figure}[H]\centering
	\includesvg[width=1.0\textwidth, inkscapelatex=true]{Pictures/ex2_result}
	\caption{Participant-selected cycle rates corresponding to virtual surface textures}\label{fig:ex2_results}
\end{figure}

In the case of the grass texture, the responses leaned toward a cycle rate of 0.5. Four out of six participants selected the medium cycle rate, while the remaining two chose 0.2. Similar to the brick condition, no participants opted for the 1.0 cycle rate for grass. This suggests that participants generally associate slower or moderate rhythmic feedback with irregular or soft textures. The preference for the 0.5 cycle rate indicates that grass, which is visually perceived as a semi-rough but flexible material, aligns best with medium vibration rhythms. This offers a balance between perceptibility and comfort.

For the marble texture, all six participants unanimously agreed that a cycle rate of 1.0 (high cycle rate) was most effective for simulating marble texture. This strong consensus suggests that fast, smooth, and continuous vibration patterns are best suited to represent smooth, hard surfaces like polished marble. This finding aligns with established tactile perception theories, which propose that dense and uninterrupted vibrational cues effectively mimic the sensation of smooth, featureless surfaces.


%%%%%%%%%%%%%%%%%%%%%%%%%%%%%%%%%%%%%%
\section{Post-Experiment Evaluation}
Alongside performance measurements, participants filled out a 12-item post-experiment questionnaire designed to evaluate their subjective impressions of virtual hand embodiment, tactile realism, and the overall quality of the VR experience. Fig.~\ref{fig:questionnaire_results} displays the summarized mean scores for all questions. This evaluation offered valuable insights into participants' perceptions of the system's effectiveness in providing immersive and tactilely enriched VR interactions.

\begin{figure}[H]\centering
	\includesvg[width=1.0\textwidth, inkscapelatex=true]{Pictures/question_result}
	\caption{Participants' accuracy in identifying different vibration cycle rates.}\label{fig:questionnaire_results}
\end{figure}

The highest agreement was observed for Q10 (“I had the sensation that I could feel the texture of objects through the virtual hand”), which reached a mean score of 5.33 out of 7, indicating strong perceived effectiveness of the vibration-based texture feedback. Similarly, Q7 (“When I touched the virtual plane, it felt as if my real hand was also being touched”) and Q6 (“The tactile sensations I perceived felt naturally aligned with the virtual hand”) received high ratings (5.00 and 4.83, respectively), supporting the conclusion that the glove effectively conveyed realistic tactile sensations during virtual interactions.

Participants reported generally positive experiences regarding immersion (Q12, mean = 4.83) and tactile attribution (Q2, mean = 4.83). This indicates that the system effectively contributed to engagement and sensory alignment within the virtual environment. There was moderate agreement on the aspects of control fidelity and perceived response accuracy, with Q4 (control of the virtual hand) scoring 4.17 and Q11 (mismatch awareness) also receiving a score of 4.17. These scores suggest that while the movement synchronization between real and virtual hands is acceptable, there is room for improvement.

However, the scores related to embodiment were lower, particularly for the following statements: Q1 (“I felt as if the virtual hands were my own”; score: 3.33), Q5 (“The virtual hand was naturally connected to my body”; score: 3.17), and Q8 (“Movements felt natural”; score: 3.17). These responses indicate a diminished sense of ownership and naturalness of movement, which aligns with the current limitations of the system regarding full-hand tracking and the responsiveness of the virtual hands.


%%%%%%%%%%%%%%%%%%%%%%%%%%%%%%%%%%%%%%
\section{General Discussion}  

This section presents a detailed statistical analysis of the subjective questionnaire data, organized into meaningful categories to assess the significance of participants' perceptions. To prepare the data for analysis according to methodological requirements, all individual responses for questions within each category were combined. A Wilcoxon Signed-Rank test was then conducted on these aggregated responses for each of the three categories, comparing them to a neutral point of 4.0. The detailed statistical results are summarized in Table~\ref{tab:wilcoxon_category_results}.

The analysis of the `Hand Ownership and Control' category, which comprised questions like Q1 ('I felt as if the virtual hands were my own'), Q3 ('My movements were accurately replicated by the virtual hand'), Q4 ('I felt I had control over the virtual hand'), Q5 ('The virtual hand was naturally connected to my body'), and Q8 ('Movements felt natural'), revealed a statistically significant finding. With a median of 3.0 and based on 30 pooled responses, this category was found to be significantly lower than the neutral point ($W=9.0$, $p=0.000275$). This indicates a notable tendency among participants to rate their sense of virtual hand ownership and control below a neutral perception.

Lastly, the `Immersion' category, which includes Question 11 ('I felt a mismatch between my real hand and the virtual hand') and Question 12 ('Overall, I felt immersed in the virtual environment'), did not demonstrate a statistically significant difference from the neutral point. The median score was 4.0, based on 12 pooled responses ($W=0.0$,  $p=0.102470$). This suggests that participants' overall sense of immersion and their perception of mismatch were not significantly different from a neutral perspective, indicating that the system did not create a notably strong sense of deep immersion.

\begin{table}[H]
    \centering
    \caption{Wilcoxon Signed-Rank Test Results for Questionnaire Categories}
    \label{tab:wilcoxon_category_results}
    \begin{tabular}{@{} p{2.5cm} c c c c S[table-format=1.6] l p{1.5cm} @{}}
        \toprule
        \textbf{Category} & \textbf{N} & \textbf{Median} & \textbf{IQR} & \textbf{Test Stat (W)} & \textbf{\textit{p}-value} & \textbf{Sig.} & \textbf{Conclusion} \\
        \midrule
        Hand Ownership \\ and Control & 30 & 3.0 & 3.00 - 4.00 & 9.0 & 0.000275 & *** & Sig. lower  \\
        \midrule
        Realistic and Aligned \\ Tactile Sensations Felt & 30 & 5.0 & 4.00 - 5.00 & 10.0 & 0.000041 & *** & Sig. higher  \\
        \midrule
        Immersion & 12 & 4.0 & 4.00 - 4.25 & 0.0 & 0.102470 & & No diff. \\
        \bottomrule
    \end{tabular}
    \caption*{Note: Significance levels: $^* p < 0.05$, $^{**} p < 0.01$, $^{***} p < 0.001$. The `N' value represents the total count of individual questionnaire responses comprising each category (number of items in category $\times$ number of participants). Test Statistic (W) is the sum of ranks of the less frequent sign among non-zero differences; it is 0.0 when all non-zero differences have the same sign. ``No diff." indicates no statistically significant difference from the neutral point of 4.0.}
\end{table}

Furthermore, qualitative feedback from some participants revealed a significant practical limitation: the fixed glove size caused discomfort and poor fit for certain individuals. This ergonomic issue is likely a factor in the non-significant results related to embodiment and naturalness of movement. An improper fit can disrupt sensory alignment and diminish the sense of virtual hand ownership. To address this limitation, future iterations of the system will focus on creating adjustable designs. The goal is to enhance comfort and optimize the effectiveness of tactile feedback, ultimately providing more realistic and immersive virtual reality experiences.

In summary, the subjective evaluation provided valuable insights into the overall user experience. The system showed a statistically significant improvement in the category of ``Realistic and Aligned Tactile Sensations Felt," confirming its effectiveness in delivering realistic and well-aligned tactile feedback. However, participants' perceptions of other important dimensions—such as embodiment, naturalness of movement, and overall immersion—were generally positive or moderate but did not reach statistical significance compared to the neutral baseline. These findings suggest that while the current prototype lays a promising foundation for tactile interaction in virtual reality, further refinement is necessary to consistently enhance subjective experiences across multiple aspects of presence and embodiment.

% The results indicate a strong relationship between vibration cycle rate and the perceived realism of textures. Participants consistently linked higher cycle rates to smooth textures, while lower to medium cycle rates were associated with rough or irregular surfaces. The absence of conflicting responses—such as no participants selecting high cycle rates for materials like brick or grass—further strengthens the reliability and consistency of vibration rhythm perception. Notably, this level of differentiation was achieved without altering the vibration amplitude, demonstrating that modulation of the cycle rate alone provides clear and intuitive tactile cues that convey distinct virtual material properties. 

% The simplicity of the system design, combined with effective texture differentiation, highlights the suitability of this approach for lightweight, real-time haptic feedback applications in virtual reality.

% Together, the experimental performance metrics and subjective questionnaire results show that the proposed PWM-based tactile feedback method significantly improves users' ability to distinguish and realistically perceive virtual textures, thereby enhancing the overall sense of immersion in virtual environments. These findings are consistent with previous literature~\cite{10.1145/3025453.3025812,10.1007/s00542-023-05486-x}.  that emphasizes the importance of tactile granularity and frequency modulation for realistic texture rendering. Moving forward, efforts should focus on strengthening the sense of embodiment, particularly by improving hand tracking fidelity and further optimizing vibration characteristics to accommodate a broader range of virtual materials. This will help achieve a more cohesive and immersive VR experience.

% The statistical analysis indicated that only one item, Q10 (“I had the sensation that I could feel the texture of objects through the virtual hand”), produced a statistically significant result. With a median score of 5.0, Q10 was notably higher than the neutral point ($W = 0.0$, $p = 0.031$). This outcome supports the effectiveness of the vibration-based texture feedback, demonstrating that participants were able to perceive virtual textures through the haptic glove with statistical significance. This finding underscores a key success in enhancing tactile realism within the system.

% In contrast, the Wilcoxon Signed-Rank test did not find any statistically significant differences from the neutral point of 4.0 for the remaining questionnaire items. This includes items that had high median scores, such as Q7 (“When I touched the virtual plane, it felt as if my real hand was also being touched,” median = 5.0, $p = 0.062$) and Q6 (“The tactile sensations I perceived felt naturally aligned with the virtual hand,” median = 5.0, $p = 0.062$).

% Similarly, items related to overall positive experiences, control fidelity, and immersion—such as Q2 (median = 5.0, $p = 0.062$), Q4 (median = 4.0, $p = 1.000$), and Q12 (median = 4.0, $p = 0.500$)—did not produce statistically significant results. 

% A noteworthy observation pertains to question 11 (“I felt a mismatch between my real hand and the virtual hand”), which was phrased negatively. In this context, higher scores indicate a greater sense of mismatch. The analysis showed a median score of 4.0 ($W = 0.0$, $p = 1.000$), indicating no statistically significant deviation from the neutral point. 

% Finally, for the embodiment-related items—Q1 (“I felt as if the virtual hands were my own”; median = 3.0, $p = 0.125$), Q5 (“The virtual hand was naturally connected to my body”; median = 3.0, $p = 0.062$), and Q8 (“Movements felt natural”; median = 3.0, $p = 0.062$)—the median scores were descriptively lower than neutral, yet none reached statistical significance. 


% \begin{table}[H]
%     \centering
%     \caption{Wilcoxon Signed-Rank Test Results for Questionnaire Items}
%     \label{tab:wilcoxon_results_final_exact}
%     \begin{tabular}{@{} l c c c c S[table-format=1.5] l p{1.5cm} @{}} % Adjusted S[table-format] for p-value precision
%         \toprule
%         \textbf{Question} & \textbf{N} & \textbf{Median} & \textbf{IQR} & \textbf{Test Stat (W)} & \textbf{\textit{p}-value} & \textbf{Sig.} & \textbf{Conclusion} \\
%         \midrule
%         Q1  & 6 & 3.0 & 3.00 - 3.75 & 0.0 & 0.1250 & & No diff. \\ 
%         \midrule
%         Q2  & 6 & 5.0 & 5.00 - 5.00 & 0.0 & 0.0625 & & No diff. \\ 
%         \midrule
%         Q3  & 6 & 4.0 & 3.25 - 4.00 & 0.0 & 0.5000 & & No diff. \\ 
%         \midrule
%         Q4  & 6 & 4.0 & 4.00 - 4.00 & 0.0 & 1.0000 & & No diff. \\ 
%         \midrule
%         Q5  & 6 & 3.0 & 3.00 - 3.00 & 0.0 & 0.0625 & & No diff. \\ 
%         \midrule
%         Q6  & 6 & 5.0 & 5.00 - 5.00 & 0.0 & 0.0625 & & No diff. \\ 
%         \midrule
%         Q7  & 6 & 5.0 & 5.00 - 5.00 & 0.0 & 0.0625 & & No diff. \\ 
%         \midrule
%         Q8  & 6 & 3.0 & 3.00 - 3.00 & 0.0 & 0.0625 & & No diff. \\ 
%         \midrule
%         Q9  & 6 & 4.0 & 4.00 - 4.00 & 0.0 & 1.0000 & & No diff. \\ 
%         \midrule
%         Q10 & 6 & 5.0 & 5.00 - 5.75 & 0.0 & 0.0313 & * & Sig. higher than Neutral. \\ 
%         \midrule
%         Q11 & 6 & 4.0 & 4.00 - 4.00 & 0.0 & 1.0000 & & No diff. \\ 
%         \midrule
%         Q12 & 6 & 4.0 & 4.00 - 4.75 & 0.0 & 0.5000 & & No diff. \\ 
%         \bottomrule
%     \end{tabular}
%     \caption*{Note: Significance levels: $^* p < 0.05$, $^{**} p < 0.01$, $^{***} p < 0.001$. Test Statistic (W) is the sum of ranks of the less frequent sign among non-zero differences; it is 0.0 when all non-zero differences have the same sign. ``No diff." indicates no statistically significant difference from the neutral point of 4.0.}
% \end{table}

% In summary, the subjective evaluation provided valuable insights into the overall user experience. The system demonstrated a statistically significant improvement in tactile texture perception (Q10), confirming its effectiveness in delivering realistic texture feedback. However, participants' perceptions of other key dimensions—such as embodiment, naturalness of movement, and overall immersion—were often described as positive or moderate, but did not reach statistical significance when compared to a neutral baseline. These results indicate that, while the current prototype establishes a promising foundation for tactile interaction in virtual reality, further improvements are needed to consistently enhance subjective experiences across multiple dimensions of presence and embodiment.
% Furthermore, qualitative feedback from some participants revealed a significant practical limitation: the fixed glove size caused discomfort and poor fit for certain individuals. This ergonomic issue is likely a factor in the non-significant results related to embodiment and naturalness of movement. An improper fit can disrupt sensory alignment and diminish the sense of virtual hand ownership. To address this limitation, future iterations of the system will focus on creating adjustable designs. The goal is to enhance comfort and optimize the effectiveness of tactile feedback, ultimately providing more realistic and immersive virtual reality experiences.
