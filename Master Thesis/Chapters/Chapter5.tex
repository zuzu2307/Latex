\chapter{Result and Discussion} % Main chapter title

\label{Chapter5} % Change X to a consecutive number; for referencing this chapter elsewhere, use \ref{ChapterX}


The following sections outline the experimental evaluation of the proposed system, which consists of two main experiments and a subjective evaluation conducted afterward. The first experiment assesses participants' ability to differentiate between three distinct cycle rates, while the second experiment investigates how these cycle rates affect the perceived realism of various virtual textures. Both experiments utilize objective classification accuracy as well as subjective evaluation scores to validate the effectiveness of the proposed method.

%----------------------------------------------------------------------------------------
%	SECTION 1
%----------------------------------------------------------------------------------------
\section{Experiment 1's Results and Discussion}
The performance of participants in identifying the three vibration cycle rates—0.2, 0.5, and 1.0—is summarized in Fig.~\ref{fig:ex1_results} and Fig.~\ref{fig:C_W_ex1}, which shows the overall classification accuracy across all trials. Each cycle rate had a total of 30 trials (six participants with five repetitions each). The highest classification accuracy was recorded at the 1.0 cycle rate, where participants correctly identified the vibration pattern in 26 out of 30 trials, resulting in an accuracy of 86.7\%. This suggests that higher-frequency vibration rhythms with shorter pauses are more easily perceived and classified by users, likely due to their more continuous and distinct tactile sensation.

\begin{figure}[H]\centering
	\includesvg[width=1.0\textwidth, inkscapelatex=true]{Pictures/ex1_result}
	\caption{Participants' accuracy in identifying different vibration cycle rates.}\label{fig:ex1_results}
\end{figure}

\begin{figure}[H]\centering
	\includesvg[width=0.8\textwidth, inkscapelatex=true]{Pictures/C_W_Chart}
	\caption{Participants' correct and wrong answers in identifying different vibration cycle rates.}\label{fig:C_W_ex1}
\end{figure}


At a cycle rate of 0.5, accuracy slightly decreased to 24 out of 30 trials (80\%). The lowest accuracy was observed at a cycle rate of 0.2, with participants achieving 23 out of 30 correct identifications (76.7\%). These results indicate a trend in which participant accuracy declines as the vibration rhythm slows down, leading to a more intermittent tactile sensation. %This finding aligns with previous research in haptic perception, which has shown that higher-frequency or more intense vibrotactile cues are generally easier to detect and discriminate due to stronger stimulation of tactile receptors.

The moderate reduction in accuracy at lower cycle rates may be attributed to the diminished temporal density of tactile feedback, making it more difficult for participants to form clear perceptual distinctions, especially when the vibration intervals become more sparse. Notably, even at the lowest cycle rate, participants maintained a classification rate well above chance level (33.3\%), indicating that the PWM cycle rate modulation approach provides reliably distinguishable tactile patterns.

Overall, these findings validate the effectiveness of the proposed vibrotactile feedback method, which utilizes pulse-width modulation to control vibration cycle rates, in delivering perceivable differences in vibrotactile rhythms. The results demonstrate that participants can accurately classify distinct cycle rates, particularly in the higher rhythm ranges, supporting the system’s suitability for conveying virtual texture roughness and bumpiness through simple vibration control strategies.

%%%%%%%%%%%%%%%%%%%%%%%%%%%%%%%%%%%%%%
\section{Experiment 2's Results and Discussion}
Participants' preferences for associating vibration cycle rates with the virtual textures—brick, grass, and marble—are illustrated in Fig.~\ref{fig:ex2_results}. The distribution of selections demonstrates how participants mapped specific vibration rhythms to different surface types. For the brick texture, responses were equally divided between 0.2 (low cycle rate) and 0.5 (medium cycle rate), with three participants choosing each option. This split suggests that participants found both slower and moderate rhythmic vibrations to be appropriate representations of the rough and uneven characteristics of brick surfaces. Notably, no participants selected the 1.0, indicating a shared perception that fast, continuous vibrations were not suitable for representing rough textures.

\begin{figure}[H]\centering
	\includesvg[width=1.0\textwidth, inkscapelatex=true]{Pictures/ex2_result}
	\caption{Participant-selected cycle rates corresponding to virtual surface textures}\label{fig:ex2_results}
\end{figure}

In the case of the grass texture, the responses leaned toward a cycle rate of 0.5. Four out of six participants selected the medium cycle rate, while the remaining two chose 0.2. Similar to the brick condition, no participants opted for the 1.0 cycle rate for grass. This suggests that participants generally associate slower or moderate rhythmic feedback with irregular or soft textures. The preference for the 0.5 cycle rate indicates that grass, which is visually perceived as a semi-rough but flexible material, aligns best with medium vibration rhythms. This offers a balance between perceptibility and comfort.

For the marble texture, all six participants unanimously agreed that a cycle rate of 1.0 (high cycle rate) was most effective for simulating marble texture. This strong consensus suggests that fast, smooth, and continuous vibration patterns are best suited to represent smooth, hard surfaces like polished marble. This finding aligns with established tactile perception theories, which propose that dense and uninterrupted vibrational cues effectively mimic the sensation of smooth, featureless surfaces.


%%%%%%%%%%%%%%%%%%%%%%%%%%%%%%%%%%%%%%
\section{Post-Experiment Evaluation}
Alongside performance measurements, participants filled out a 12-item post-experiment questionnaire designed to evaluate their subjective impressions of virtual hand embodiment, tactile realism, and the overall quality of the VR experience. Fig.~\ref{fig:questionnaire_results} displays the summarized mean scores for all questions. This evaluation offered valuable insights into participants' perceptions of the system's effectiveness in providing immersive and tactilely enriched VR interactions.

\begin{figure}[H]\centering
	\includesvg[width=1.0\textwidth, inkscapelatex=true]{Pictures/question_result}
	\caption{Participants' accuracy in identifying different vibration cycle rates.}\label{fig:questionnaire_results}
\end{figure}

The highest agreement was observed for Q10 (“I had the sensation that I could feel the texture of objects through the virtual hand”), which reached a mean score of 5.33 out of 7, indicating strong perceived effectiveness of the vibration-based texture feedback. Similarly, Q7 (“When I touched the virtual plane, it felt as if my real hand was also being touched”) and Q6 (“The tactile sensations I perceived felt naturally aligned with the virtual hand”) received high ratings (5.00 and 4.83, respectively), supporting the conclusion that the glove effectively conveyed realistic tactile sensations during virtual interactions.

Participants reported generally positive experiences regarding immersion (Q12, mean = 4.83) and tactile attribution (Q2, mean = 4.83). This indicates that the system effectively contributed to engagement and sensory alignment within the virtual environment. There was moderate agreement on the aspects of control fidelity and perceived response accuracy, with Q4 (control of the virtual hand) scoring 4.17 and Q11 (mismatch awareness) also receiving a score of 4.17. These scores suggest that while the movement synchronization between real and virtual hands is acceptable, there is room for improvement.

However, the scores related to embodiment were lower, particularly for the following statements: Q1 (“I felt as if the virtual hands were my own”; score: 3.33), Q5 (“The virtual hand was naturally connected to my body”; score: 3.17), and Q8 (“Movements felt natural”; score: 3.17). These responses indicate a diminished sense of ownership and naturalness of movement, which aligns with the current limitations of the system regarding full-hand tracking and the responsiveness of the virtual hands.

% \subsection{General Discussion}  
% The results indicate a strong relationship between vibration cycle rate and the perceived realism of textures. Participants consistently linked higher cycle rates to smooth textures, while lower to medium cycle rates were associated with rough or irregular surfaces. The absence of conflicting responses—such as no participants selecting high cycle rates for materials like brick or grass—further strengthens the reliability and consistency of vibration rhythm perception. Notably, this level of differentiation was achieved without altering the vibration amplitude, demonstrating that modulation of the cycle rate alone provides clear and intuitive tactile cues that convey distinct virtual material properties. 

% The simplicity of the system design, combined with effective texture differentiation, highlights the suitability of this approach for lightweight, real-time haptic feedback applications in virtual reality.

% Together, the experimental performance metrics and subjective questionnaire results show that the proposed PWM-based tactile feedback method significantly improves users' ability to distinguish and realistically perceive virtual textures, thereby enhancing the overall sense of immersion in virtual environments. These findings are consistent with previous literature~\cite{10.1145/3025453.3025812,10.1007/s00542-023-05486-x}.  that emphasizes the importance of tactile granularity and frequency modulation for realistic texture rendering. Moving forward, efforts should focus on strengthening the sense of embodiment, particularly by improving hand tracking fidelity and further optimizing vibration characteristics to accommodate a broader range of virtual materials. This will help achieve a more cohesive and immersive VR experience.
