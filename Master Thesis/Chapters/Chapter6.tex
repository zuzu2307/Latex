\chapter{Conclusion} % Main chapter title

\label{Chapter6}

This study investigated methods to enhance immersion in virtual reality experiences through the development of a wearable haptic glove. The glove is equipped with flex sensors, an IMU (MPU-6050), and a coin-type vibration motor. I utilized PWM-based vibrotactile feedback to examine participants' ability to differentiate between various vibration cycle rates and to assess how these tactile signals influenced the perceived realism of virtual textures.

In two structured experiments, the results indicated that participants could reliably distinguish between textures based on their smoothness. Higher accuracy was observed at higher cycle rates, especially when identifying differences among textures with varying levels of roughness. Subjective evaluations were analyzed using Wilcoxon Signed-Rank tests on composite category scores, revealing a statistically significant improvement in the category of `Realistic and Aligned Tactile Sensations Felt.' However, perceptions related to the categories of `Hand Ownership and Control' and `Immersion,' while often showing favorable trends, did not reach statistical significance when compared to a neutral baseline. A significant factor contributing to these challenges, as highlighted by participant feedback, was the poor fit of the fixed-size glove.

Building on these findings, future work will focus on optimizing feedback parameters, such as vibration cycle rates and duty cycles, to achieve more accurate and nuanced texture rendering. Additionally, the system will be expanded to include a wider variety of virtual textures, enhancing realism across different surface types. 

Furthermore, I will increase the number of participants in subsequent experiments to improve the statistical robustness and generalizability of the results. I also aim to implement adaptive feedback algorithms capable of dynamically adjusting haptic parameters based on real-time user interactions, thereby further enhancing the immersion and interactivity of VR applications.





