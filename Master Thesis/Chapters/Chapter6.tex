\chapter{Conclusion} % Main chapter title

\label{Chapter6}

This study investigated methods to enhance immersion in virtual reality experiences through the development of a wearable haptic glove. The glove is equipped with flex sensors, an IMU (MPU-6050), and a coin-type vibration motor. We utilized PWM-based vibrotactile feedback to examine participants' ability to differentiate between various vibration cycle rates and to assess how these tactile signals influenced the perceived realism of virtual textures.

In two structured experiments, the results showed that participants could reliably distinguish textures based on smoothness, with higher accuracy observed at faster cycle rates, particularly when identifying differences among textures with varying levels of roughness. Subjective evaluations further confirmed enhancements in tactile realism, immersion, and user satisfaction. However, some limitations were noted regarding virtual embodiment and the perception of hand ownership.

Building on these findings, future work will focus on optimizing feedback parameters, such as vibration cycle rates and duty cycles, to achieve more accurate and nuanced texture rendering. Additionally, the system will be expanded to include a wider variety of virtual textures, enhancing realism across different surface types. To address limitations in embodiment, we plan to refine sensor placement, improve calibration accuracy, and consider the integration of additional haptic actuators to provide richer multi-point feedback. 

Furthermore, we will increase the number of participants in subsequent experiments to improve the statistical robustness and generalizability of the results. We also aim to implement adaptive feedback algorithms capable of dynamically adjusting haptic parameters based on real-time user interactions, thereby further enhancing the immersion and interactivity of VR applications.





